% \documentclass{article}
% \usepackage{amsmath}


% \title{COMP2120 Tutorial 2}
% \author{YUAN Wenxuan (UID: 3036292740)}
% \date{\today}

% \begin{document}

% \subsection*{Tutorial Exercise}
% A machine uses 36-bit word to represent single-precision floating point numbers as follows:

% \begin{table}[!ht]
%     \centering
%     \begin{tabular}{|c|c|c|}
%         \hline
%         S & 11-bit exponent (E) & 24-bit Significand (M) \\ \hline
%     \end{tabular}
% \end{table}

% The value presented is given by $(-1)^{S} \times 1.M \times 2^{E-1023}$.

% (a) Write down the bit pattern corresponding to the value 7.375 \\
% (b) Write down the value corresponding to the bit pattern C05D00000 \\
% (c) What is the largest positive number that can be represented, assuming there is no pattern with special meaning in the representation? \\
% (d) What is the smallest positive number other than 0 that can be represented, assuming no pattern with special meaning?

% \end{document}

\documentclass{exam}
\usepackage{amsmath}

\title{COMP2120 Tutorial Exercise}
\author{YUAN Wenxuan (UID: 3036292740)}
\date{\today}

\begin{document}
\maketitle
% \begin{questions}
A machine uses 36-bit word to represent single-precision floating point numbers as follows:
\begin{table}[!ht]
    \centering
    \begin{tabular}{|c|c|c|}
        \hline
        S & 11-bit exponent (E) & 24-bit Significand (M) \\ \hline
    \end{tabular}
\end{table}


The value presented is given by $(-1)^{S} \times 1.M \times 2^{E-1023}$.

\begin{parts}
    \part Write down the bit pattern corresponding to the value 7.375
    \part Write down the value corresponding to the bit pattern C05D00000
    \part What is the largest positive number that can be represented, assuming there is no pattern with special meaning in the representation?
    \part What is the smallest positive number other than 0 that can be represented, assuming no pattern with special meaning?
\end{parts}
% \end{questions}

\subsection*{(a)}
$7.375_{10} = 111.011_{2} = 1.11011_{2} \times 2^{2}$
Thus, $S = 0,\ E = 1025 = 100\ 0000\ 0001_{2},\ M = (1101\ 1000\ 0000\ \dots)_{2}$
Hence, the bit pattern is $0100\ 0000\ 0001\ 1101\ 1000\ 0000\ 0000\ 0000\ 0000_{2} = 401D80000_{16}$

\subsection*{(b)}
$C05D00000_{16} = 1100\ 0000\ 0101\ 1101\ 0000\ 0000\ 0000\ 0000\ 0000_{2}$ \\
Thus, $S = 1, E = 100\ 0000\ 0101_{2} = 1029, M = (1101\ 0000\ \dots)_{2}, 1.M = 1.1101_{2} = 1.8125$ \\
Hence, the value is $-1 \times 2^{1029 - 1023} \times 1.8125 = -116.0$

\subsection*{(c)}
The largest positive number is $2^{1024} - 2^{1000}$ when $S = 0, E = 111\ 1111\ 1111, M = 1111\ 1111\ \dots\ 1111$.

\subsection*{(d)}
The smallest positive number is $2^{-1023}$ when $S = 0, E = 000\ 0000\ \dots\ 0000, M = 0000\ 0000\ \dots\ 0000$.

\end{document}
