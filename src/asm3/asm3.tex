\documentclass[answers]{exam}
\usepackage{amsmath}
\usepackage{minted}

\title{COMP2120 Assignment 3}
\author{YUAN Wenxuan (UID: 3036292740)}
\date{\today}

\begin{document}
\maketitle

\begin{questions}
    \question \textbf{2-way Set Associative Cache Memory} \\
    Consider a hypothetical machine with 1K words of cache memory. They are in two-way set associative organization, with cache block size of 128 words, using LRU replacement algorithm.
    Suppose the hit time is 10ns, the time to transfer the first word from main memory to cache is 60ns, while subsequent words require 12ns/word. \\
    Consider the following read pattern (in blocks of 128 words, and block id starts from 0):
    \begin{listing}[htb]
        \centering
        \begin{minted}{c}
        1 2 3 5 6 2 3 4 9 10 11 6 3 6 1 7 8 4 5 9 11 1 2 4 5 12 13 14 15 13 14
        \end{minted}
    \end{listing} \\
    and assume each block contains an average of 48 references.

    \begin{parts}
        \part What is the cache miss penalty (i.e., time to transfer one block of data from main memory to cache memory)?
        \begin{solution}
            The cache miss penalty is $60 + 12 \times 127 = 1584$ns.
        \end{solution}

        \part Write down the content of the cache memory (for all the blocks) at the end of the memory references, assuming that the cache is empty at the beginning.
        \begin{solution}
            \begin{tabular}{|c|c|c|c|}
                \hline
                Set 0 & Set 1 & Set 2 & Set 3 \\
                \hline
                4     & 13    & 14    & 11    \\
                \hline
                12    & 5     & 2     & 15    \\
                \hline
            \end{tabular}
        \end{solution}

        \part Write down the number of cache misses (the first reading of a block is also considered a miss), and the cache hit rate.
        \begin{solution}
            The number of cache misses is 22, thus the cache hit rate is $1 - \frac{22}{31 \times 48} = 98.5215\% $
        \end{solution}

        \part Calculate the average memory access time.
        \begin{solution}
            The average memory access time is $10 + (1 - 98.5215\%) \times 1584 = 33.4194$ns.
        \end{solution}
    \end{parts}

    \newpage
    \question \textbf{Direct-Mapped Cache Memory} \\
    Redo Question 1 if the cache is the same, but in direct-mapped organization, and the cache hit time is 9ns instead.

    \begin{parts}
        \part What is the cache miss penalty (i.e., time to transfer one block of data from main memory to cache memory)?

        \begin{solution}
            The cache miss penalty is $60 + 12 \times 127 = 1584$ns.
        \end{solution}

        \part Write down the content of the cache memory (for all the blocks) at the end of the memory references, assuming that the cache is empty at the beginning.
        \begin{solution}
            \begin{tabular}{|c|c|c|c|c|c|c|c|c|}
                \hline
                Block 0 & Block 1 & Block 2 & Block 3 & Block 4 & Block 5 & Block 6 & Block 7 \\
                \hline
                8       & 1       & 2       & 11      & 12      & 13      & 14      & 15      \\
                \hline
            \end{tabular}
        \end{solution}

        \part Write down the number of cache misses (the first reading of a block is also considered a miss), and the cache hit rate.
        % TODO: Recalculate the expression with scientific calculator.
        \begin{solution}
            The number of cache misses is 21, thus the cache hit rate is $1 - \frac{21}{31 \times 48} = 98.5887\% $
        \end{solution}

        % TODO: Recalculate the expression with scientific calculator.
        \part Calculate the average memory access time.
        \begin{solution}
            The average memory access time is $9 + (1 - 98.5887\%) \times 1584 = 31.3548$ns.
        \end{solution}
    \end{parts}

    \question \textbf{Hard Disk access} \\
    Consider a Hard Disk with an average seek time of 12ms and rotation speed of 7200rpm, and an average number of 500 sectors per track. Assume negligible transfer time. \\

    \begin{parts}
        \part What is the average rotation latency?
        \begin{solution}
            The average rotation latency is $\frac{60 \times 1000}{7200} \times \frac{1}{2} = 4.1667$ms.
        \end{solution}

        \part What is the average time to rotate for 1 sector?
        \begin{solution}
            The average time is $\frac{1}{500} \times \frac{1000}{120} = 0.0167$ms.
        \end{solution}

        \part Consider the access of 5 sectors. Calculate the time required (ignoring transfer time, but including rotation time for reading a sector) if
        \begin{subparts}
            % TODO: Recalculate the expression with scientific calculator.
            \subpart The sectors are consecutive in the same track.
            \begin{solution}
                The required time is $12 + 4.1667 + 5 \times 0.0167 = 16.25$ms.
            \end{solution}

            \subpart The sectors are scattered in various places in the HDD.
            \begin{solution}
                The required time is $5 \times (12 + 4.1667 + 0.0167) = 80.9167$ms.
            \end{solution}
        \end{subparts}
    \end{parts}

\end{questions}

\end{document}