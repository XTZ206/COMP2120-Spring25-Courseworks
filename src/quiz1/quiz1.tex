\documentclass[answers]{exam}
\usepackage{amsmath}

\usepackage{listings}
\usepackage{minted}


\begin{document}

\begin{center}
    \textbf{The University of Hong Kong} \\
    \textbf{Department of Computer Science} \\
    \textbf{COMP2120 A/B Computer organization}
\end{center}
\date{20/21 March 2025}

\newpage
\begin{questions}

    \question
    \begin{parts}
        \part
        \begin{subparts}
            \subpart[2] Write down the 8-bit two's complement representation of the numbers 37 and -24.

            \begin{solution}
                37: 00100101, -24: 11101000
            \end{solution}

            \subpart[3] Write down the mapping function $f$ for converting a bit pattern $P$ (represented as an unsigned integer) to the represented value V in both two's complement representation and excess-$k$ representation.
            $ V = f(P) $
            \begin{solution}
                Two's complement: $ V = P - 2^{n-1} (\text{if $V$ is negative})$, Excess-$k$: $ V = P - k $
            \end{solution}
        \end{subparts}

        \newpage

        \part Consider a 32-bit floating point representation with a sign bit $S$, an exponent $E$ (biased, 12 bits), and a significand $f$ (19 bits). No need to consider IEEE special patterns. The value is \\
        \begin{table}[h]
            \centering
            \begin{tabular}{|c|c|c|}
                \hline
                $S$ & 12-bit exponent $E$ & 19-bit significand $f$ \\
                \hline
            \end{tabular}
        \end{table}
        \begin{subparts}
            \subpart[1] Write down the value of b.
            \begin{solution}
                $b = 2^{12-1}-1 = 2047$
            \end{solution}
            \subpart[1] Write down the largest positive number that can be represented.
            \begin{solution}
                When $S = 0, E = 111\dots11, f=111\dots11$, \\
                V is the largest positive number that can be represented: $2^{2049}-2^{2029}$.
            \end{solution}
            \subpart[3] Write down the bit pattern representing the value 30.625
            \begin{solution}
                $30.625_{10} = 11110.101_{2} = 1.1110101 \times 10 ^ {4}$ \\
                Thus, $f = 1110\ 1010\ 0000\ \dots 0000$, $E = 2051 = 1000\ 0000\ 0011$, $S = 0$. \\
                The bit pattern is $0100\ 0000\ 0001\ 1111\ 0101\ 0000\ 0000\ 0000$ = \verb|401F5000|.
            \end{solution}
        \end{subparts}
    \end{parts}

    \newpage

    \question
    \begin{parts}
        \part[3] Consider the following function,
        \begin{listing}[htb]
            \centering
            \begin{minted}{c}
            int i, j, sum = 0;
                for (j = 0; j < M; j++)
                    for (i = 0; i < N; i++)
                        sum += a[i][j];
            return sum;
        \end{minted}
        \end{listing}
        \verb*|a[][]| are stired in memory with the following addresses.
        \begin{table}[htb]
            \centering
            \begin{tabular}{ccccccc}
                \hline
                Address  & 0        & 4        & 8        & 12       & 16       & 20       \\
                Contents & $a_{00}$ & $a_{01}$ & $a_{02}$ & $a_{10}$ & $a_{11}$ & $a_{12}$ \\
                \hline
            \end{tabular}
        \end{table}

        Please write down the access order of \verb|a[][]| in the function.
        Please modify the function to have a better spatial locality.

        \begin{solution}
            Access Order: $a_{00}$, $a_{10}$, $a_{01}$, $a_{11}$, $a_{02}$, $a_{12}$. \\
            Modified Function:
            \begin{minted}{c}
            int i, j, sum = 0;
            for (i = 0; i < N; i++)
                for (j = 0; j < N; j++)
                    sum += a[i][j];
            return sum;
            \end{minted}
        \end{solution}

        \newpage

        \part Consider a hypothetical machine with 1024 words of cache memory. They are in a two-way set associative organization, with cache block size of 128 words. One word is four bytes.
        \begin{subparts}
            \subpart[1] How many blocks are there in the cache memory?
            \begin{solution}
                $ \text{\#blocks} = 1024 / 128 = 8$
            \end{solution}
            \subpart[1] How many sets are there in the cache memory?
            \begin{solution}
                $ \text{\#sets} = 8 / 2 = 4$
            \end{solution}
            \subpart[5] The memory address is byte addressable, i.e. each address is one byte. For the byte with address \verb|C912F|(hex), what are its cache set number, cache tag, and offset? Suppose this byte is stored in the cache. What are the addresses of the other bytes stored along with it?
            \begin{solution}
                \verb|C912F| = $1100\ 1001\ 0001\ 0010\ 1111$ \\
                \begin{tabular}{|c|c|}
                    \hline
                    offset    & $log_{2}(128 \times 4) = 9$ bits \\
                    cache set & $\log_{2}(4) = 2$ bits           \\
                    cache tag & $20 - 9 - 2 = 9$ bits            \\
                    \hline
                \end{tabular}
                \\
                Thus, the cache set number is $00$, the cache tag is $110010010$ and the offset is $100101111$.
                \\
                The addresses of the other bytes stored along with it are from $1100\ 1001\ 0000\ 0000\ 0000$ to $1100\ 1001\ 0001\ 1111\ 1111$,
                namely, from \verb|C9000| to \verb|C91FF| (hex).
            \end{solution}
        \end{subparts}
    \end{parts}


\end{questions}
\end{document}
