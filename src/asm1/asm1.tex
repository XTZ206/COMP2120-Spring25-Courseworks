\documentclass{article}
\usepackage{amsmath, amsfonts}

\title{COMP2120 Assignment 1}
\author{YUAN Wenxuan (UID: 3036292740)}
\date{\today}

\begin{document}

\maketitle

\subsection*{Problem 1}
$P = \bar{A}\bar{B}\bar{C} + \bar{A}BC + A\bar{B}C + AB\bar{C} + ABC$


\subsection*{Problem 2}
\subsubsection*{2(a)}
32767, -32768
\subsubsection*{2(b)}
$0000\ 0000\ 0001\ 0010_{2} = \text{0012} _{16}$  \\
$1111\ 1111\ 1110\ 1110_{2} = \text{ffee} _{16}$  \\
$0000\ 0000\ 0001\ 1001_{2} = \text{0019} _{16}$  \\
$1111\ 1111\ 1110\ 0111_{2} = \text{ffe7} _{16}$
\subsubsection*{2(c)}
18, 65518, 25, 65511
\subsubsection*{2(d)}
\begin{align*}
    0000\ 0000\ 0001\ 0010_{2} + 0000\ 0000\ 0001\ 1001_{2} & = 0000\ 0000\ 0010\ 1011_{2} = 43_{10}    \\
    0000\ 0000\ 0001\ 0010_{2} + 1111\ 1111\ 1110\ 0111_{2} & = 1111\ 1111\ 1111\ 1001_{2} = (-7)_{10}  \\
    1111\ 1111\ 1110\ 1110_{2} + 0000\ 0000\ 0001\ 1001_{2} & = 0000\ 0000\ 0000\ 0111_{2} = 7_{10}     \\
    1111\ 1111\ 1110\ 1110_{2} + 1111\ 1111\ 1110\ 0111_{2} & = 1111\ 1111\ 1101\ 0101_{2} = (-43)_{10}
\end{align*}


\newpage
\subsection*{Problem 3}
If A and B are both unsigned integers, \\
then $ 0 \le A \le 2^n - 1,\ 0 \le B \le 2^m - 1$. \\
Thus,
\begin{align*}
    A \times B & \ge 0 \times 0 = 0                \\
    A \times B & \le (2^n - 1)(2^m - 1)            \\
               & \le 2^{n + m} - 2^{n} - 2^{m} + 1 \\
               & \le 2^{n+m} - 1
\end{align*}
In such case, $A \times B$ is no more than $n + m$ bits.
\\
If A and B are 2's complement, \\
then $ -2^{n-1} \le A \le 2^{n-1} - 1,\ -2^{m-1} \le B \le 2^{m-1} - 1$. \\
Thus,
\begin{align*}
    A \times B & \ge min\{ -2^{n-1}(2^{m-1} - 1), \-2^{m-1}(2^{n-1} - 1) \} \\
               & = min\{ -2^{n+m-2} + 2^{n-1}, -2^{n+m-2} + 2^{m-1}\}       \\
               & \ge -2^{n+m-2} \ge -2^{n+m-1}                              \\
    A \times B & \le (2^{n-1} - 1)(2^{m-1} - 1)                             \\
               & \le 2^{n+m-2} - 2^{n-1} - 2{m-1} + 1                       \\
               & \le 2^{m+n-1} - 1
\end{align*}
In such case, $A \times B$ is also no more than $n + m$ bits. \\
Hence, the product $ A \times B $ is no more than $n + m$ bits.


\newpage
\subsection*{Problem 4}
Let $A = \overline{A_{n} A_{n-1} A_{n-2} \dots A_{0}}$ be the multiplicand and $B = \overline{B_{n} B_{n-1} B_{n-2} \dots B_{0}}$ be the multiplier, where $A_i, B_j\in \{0, 1\},\ i,j \in \mathbb{N}, i,j \le n$
\begin{align*}
     & A = \overline{A_{n} A_{n-1} A_{n-2} \dots A_{0}} = - A_{n} \times 2^{n} + A_{n-1} \times 2^{n-1} + \dots + A_{0} \times 2^{0} \\
     & B = \overline{B_{n} B_{n-1} B_{n-2} \dots B_{0}} = - B_{n} \times 2^{n} + B_{n-1} \times 2^{n-1} + \dots + B_{0} \times 2^{0}
\end{align*}

In the first step, we calculate the partial products of $A$ and $B$ (except the sign bit), and we get
\begin{align*}
        & A \times (B_{n-1} \times 2^{n-1} + B_{n-2} \times 2^{n-2} +\dots + B_{0} \times 2^{0})                        \\
    =\  & (A \times B_{n-1} \times 2^{n-1}) + (A \times B_{n-2} \times 2^{n-2}) + \dots + (A \times B_{0} \times 2^{0})
\end{align*}
Then, in the second step, each item of the partial product is sign extended to $2n$ bits, so that their decimal value remains. \\
If $B$ is positive, $B_{n} = 0$, then the sum of the partial products is the final result.
\begin{align*}
    A \times B & = A \times (-B_{n} \times 2^{n} + B_{n-1} \times 2^{n-1} + B_{n-2} \times 2^{n-2} +\dots + B_{0} \times 2^{0}) \\
               & = A \times (B_{n-1} \times 2^{n-1} + B_{n-2} \times 2^{n-2} +\dots + B_{0} \times 2^{0})                       \\
               & = \text{the sum of the pratial product}
\end{align*}
If $B$ is negative, $B_{n} = 1$, then an extra item is added to the partial sum.
\begin{align*}
    A \times B & = A \times (-B_{n} \times 2^{n} + B_{n-1} \times 2^{n-1} + B_{n-2} \times 2^{n-2} +\dots + B_{0} \times 2^{0}) \\
               & = A \times (B_{n-1} \times 2^{n-1} + B_{n-2} \times 2^{n-2} +\dots + B_{0} \times 2^{0}) - A \times 2^{n}      \\
               & = \text{the sum of the pratial product} - A \times 2^{n}
\end{align*}
According to the progress, the extra item is the sign extended two's complement of $A$, whose decimal value is $- A \times 2^{n}$. Thus, the final sum is consistent with the result above.

Hence, the multiplication of integers procedure in the lecture not is valid.

\newpage
\subsection*{Problem 5}
$ A = 0.4_{10} = (0.01100110011001100110\dots)_{2} $ \\
Thus, rounding to $A' = (1.1001101)_{2} \times 2^{-2} = 0.400390625$. \\
Therefore, $r = \frac{A - A'}{A} = \frac{0.4- 0.400390625}{0.4} = -\frac{1}{1024}$


\subsection*{Problem 6}
\subsubsection*{6(a)}
The largest positive value is $2^{1025} - 2^{996}$ when $ f = (111 \dots 111)_{2},\ E = 2047 $
\subsubsection*{6(b)}
The smallest positive value is $2^{-1023}$ when $ f = (000 \dots 000)_{2},\ E = 0 $

\subsubsection*{6(c)}
$15.3125_{10} = (1111.0101)_{2} = (1.1110101)_{2} \times 2^{3}$ \\
Thus, $ S = 0,\ f = 1110\ 1010\ 0000\ \dots,\ E = 1026_{10} = 100\ 0000\ 0010$ \\
The bit pattern is $0100\ 0000\ 0010\ 1110\ 1010\ 0000\ 0000\ 0000\ 0000\ 0000_{2} = \text{402ea00000} _{16}$
\subsubsection*{6(d)}
$\text{c06f800000} _{16} = 1100\ 0000\ 0110\ 1111\ 1000\ 0000\ 0000\ 0000\ 0000\ 0000_{2}$ \\
Thus, $ S = 1,\ 1.f = (1.111\ 1100\ \dots)_{2} = 1.96875,\ E = 100\ 0000\ 0110_{2} = 1030_{10}$, then the value is $ -1 \times 1.96875 \times 2^{7} = -252.0$.
\subsubsection*{6(e)}
In this case, the largest positive number that can be represented is \\ $(1 - \frac{1}{2^{23}}) \times 2^{2^15} = 2^{32769} - 2^{32745}$. \\
By assigning more bits for exponent and fewer bits for significand, the range increases, while there will be less precision.
As the total number of the numbers that can be represented is always a certain value, there always exists a tradeoff between range and precision.


\end{document}