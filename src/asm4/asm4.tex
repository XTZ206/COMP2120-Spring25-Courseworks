\documentclass[answers]{exam}
\usepackage{amsmath}
\usepackage{minted}

\title{COMP2120 Assignment 4}
\author{YUAN Wenxuan (UID: 3036292740)}
\date{\today}
\pagestyle{plain}

\begin{document}

\maketitle

\begin{questions}
    \question Hand assemble the following assembly code and put it in a program file. Run the simulator on this program. Explain what the function \mintinline{text}|SQ| does?
    \begin{listing}[htb]
        \begin{minted}{text}
            SUB     R4,R4,R4        0000H: 01040404
            LD      P1,R1           0004H: 0600ff01 00000078
            MOV     R1,R2           000CH: 05010002
            LD      P2,R3           0010H: 0600ff03 0000007c
        L:  MOV     R1,R10          0018H: 0501000a
            CALL    SQ              001CH: 0c00ff00 00000044
            ADD     R4,R11,R4       0024H: 00040b04
            ADD     R1,R2,R1        0028H: 00010201
            SUB     R3,R1,R5        002CH: 01030105
            BNZ     L               0030H: 0802ff00 00000018
            ST      R4,P            0038H: 0704ff00 00000080
            HLT                     0040H: 09000000
        
        /* Procedure to calculate ____, input is R10, output is R11 */
        /* The proc uses R12 and R13, need to save them on entry */
        /* and restore them when exit*/
        
        SQ: PUSH    R12             0044H: ....
            PUSH    R13             0048H: ....
            LD      P1,R13          004CH: ....
            SUB     R11,R11,R11     0054H:
            MOV     R10,R12         0058H:
        L2: ADD     R11,R10,R11     005CH:
            SUB     R12,R13,R12     0060H:
            BNZ     L2              0064H:
            POP     R13             006CH:
            POP     R12             0070H:
            RET                     0074H:
        P1: .WORD   1               0078H: 00000001
        P2: .WORD   A               007CH: 0000000a
        P:  .WORD                   0080H: 00000000
        \end{minted}
    \end{listing}

    \pagebreak
    \begin{solution}
        The function \mintinline{text}|SQ| calculates the square of \mintinline{text}|R10| and stores the result in \mintinline{text}|R11|. \\
        In other words, \mintinline{text}|R11| = \mintinline{text}|R10|$\times$ \mintinline{text}|R10|. \\
        The following is the hand assembled code:
        \begin{minted}{text}
0044H:  0a0c0000
0048H:  0a0d0000
004CH:  0600ff0d 00000078
0054H:  010b0b0b
0058H:  050a000c
005CH:  000b0a0b
0060H:  010c0d0c
0064H:  0802ff00 0000005c
006CH:  0b00000d
0070H:  0b00000c
0074H:  0d000000
        \end{minted}
        You can also find it in the attached file \mintinline{text}|prog1|
    \end{solution}


    \question Run the simulator in debug mode. Write down the data transfer/transformation sequences involved in the execution of the instructions \mintinline{text}|CALL| and \mintinline{text}|RET|.
    You may skip intermediate step provided by the simulator, for example the instruction fetching step should look like:
    \begin{minted}{text}
    MAR <- PC
    IR  <- mem[MAR]
    \end{minted}
    or in English, move the value of \mintinline{text}|PC| to \mintinline{text}|MAR|. Then read memory and the result(\mintinline{text}|mem[MAR]|) is moved to \mintinline{text}|IR|, i.e. just write down the source and destination of the data movement, without the paths etc.

    \begin{solution}
        data transfer/transformation sequences: \\
        In \mintinline{text}|CALL|:
        \begin{minted}{text}
MAR <- PC
MAR <- mem[MAR]
PC <- PC + 4
TEMP <- MAR
SP <- SP - 4
MAR <- SP
mem[MAR] <- PC
MAR <- TEMP
PC <- MAR        
        \end{minted}
        In \mintinline{text}|RET|:
        \begin{minted}{text}
MAR <- SP
SP <- SP + 4
PC <- mem[MAR]     
        \end{minted}
    \end{solution}

    \pagebreak
    \question Modify the program so that it will calculate the value of $1 - 2 + 3 - 4 \cdots - 8 + 9$.
    That is,
    \begin{minted}{text}
    sum = 0;
    for i = 1 to 9 do sum += sq(i)
    \end{minted}
    Where \mintinline{text}|sq(i)| return \mintinline{text}|i| when \mintinline{text}|i| is odd, otherwise return \mintinline{text}|-i|.
    Note that the original program is already a loop from 1 to 9. Just replace the function \mintinline{text}|SQ| by
    \begin{minted}{text}
    if (R10 is odd) R11 = R10;
    else R11 = 0 - R10;
    \end{minted}
    Since we don't have a \mintinline{text}|NEG| instruction, to find $-x$, we use $0 - x$.
    To check if a number $x$ is odd, just check if the rightmost is $1$. We can find $x$ \mintinline{text}|AND| $00000000 \cdots 0001$. (i.e. 1) After \mintinline{text}|AND| operation, all bits \mintinline{text}|AND|ed with 0 will be 0. If the rightmost bit is 0, then the result is 0. Otherwise the result is non-zero.
    Note that the address of \mintinline{text}|P1|, \mintinline{text}|P2| and \mintinline{text}|P| may got changed when the length of the function is changed. You may need to change the address of them in the program, e.g. in line 2
    \begin{minted}{text}
    LD  P1,R1
    \end{minted}
    you may need to find the new address of \mintinline{text}|P1|, and also in line 4 \dots

    \begin{solution}
        The following is the modified assembled code:
        \begin{minted}{text}
0000H:  01040404
0004H:  0600ff01 0000007c
000CH:  05010002
0010H:  0600ff03 00000080
0018H:  0501000a
001CH:  0c00ff00 00000044
0024H:  00040b04
0028H:  00010201
002CH:  01030105
0030H:  0802ff00 00000018
0038H:  0704ff00 00000084
0040H:  09000000
0044H:  0a0c0000
0048H:  0a0d0000
004CH:  0600ff0d 0000007c
0054H:  010b0b0b
0058H:  050a000c
005CH:  030c0d0d
0060H:  0802ff00 0000006c
0068H:  010b0c0c
006CH:  050c000b
0070H:  0b00000d
0074H:  0b00000c
0078H:  0d000000
007CH:  00000001
0080H:  0000000a
0084H:  00000000
        \end{minted}
        You can also find it in the attached file \mintinline{text}|prog3|
    \end{solution}

\end{questions}
\end{document}
